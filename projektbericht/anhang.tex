\chapter{Quellcode}
\label{anhang:quellcode}
Hier werden die wichtigsten Ausschnitte des Quellcodes gezeigt. 
\section{Skelettierung - Startup}
 \lstinputlisting
    [caption={Startup der Skelettierung. Hier werden die Spielersegmentierung und die Skelettierungsalgorithmen ausgeführt.}
       \label{lst:startup},
       captionpos=t,language=python]
 {./listing/fuerAnhang/run.py}
\section{Spielersegmentierung}
\label{anhang:segmentierung}
\lstinputlisting
    [caption={Spielersegmentierung}
       \label{lst:anhang_spielersegmentierung},
       captionpos=t,language=python]
 {./listing/fuerAnhang/player_segmentation.py}
\section{Skelettierung - Distanztransformation}
\lstinputlisting
    [caption={Skelettierung mittels Distanztransformation. Implementierung der in Abschnitt \ref{sec:distanztransformation} beschriebenen Schritte zur Extraktion eines Skeletts.}
       \label{lst:anhang_distanztransformation},
       captionpos=t,language=python]
 {./listing/fuerAnhang/skeletonization.py}
\section{Skelettierung - Thinning}
%TODO: C++ Programm und pythonWrapper
\section{Verbesserung der Skelettqualität}
\subsection{Breitensuche}
 \lstinputlisting
    [caption={Implementierung der Verbesserung der Skelettqualität mittels Breitensuche.}
       \label{lst:breitensuche},
       captionpos=t,language=python]
 {./listing/fuerAnhang/skeleton_improvement.py}
 \newpage
\subsection{Tiefensuche}
\lstinputlisting
    [caption={Verbinden der Zusammenhangskomponenten, die mit der Tiefensuche gefunden wurden.}
       \label{lst:connectcomponents},
       captionpos=t,language=python]
 {./listing/fuerAnhang/connectComponents2.py}
 
 \lstinputlisting
     [caption={Die Tiefensuche.}
        \label{lst:tiefensuche},
        captionpos=t,language=python]
  {./listing/fuerAnhang/dfs.py}
  
 \lstinputlisting
      [caption={Finden des nächstgelegenen Punktes.}
         \label{lst:findnextfeature},
         captionpos=t,language=python]
   {./listing/fuerAnhang/nextfeature.py}